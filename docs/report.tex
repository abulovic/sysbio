\documentclass[12pt,a4paper,titlepage]{article}
\usepackage{fullpage}
\usepackage{hyperref}
\usepackage[pdftex]{graphicx}
\bibliographystyle{plain}
\newcommand{\HRule}{\rule{\linewidth}{0.5mm}}
\usepackage{url}
\usepackage{amsmath}
\usepackage{enumitem}
\usepackage{float}
\usepackage{graphicx}
\usepackage{caption}
\usepackage{subcaption}
\usepackage{tabularx}
\usepackage[nottoc]{tocbibind}
\setlength{\parindent}{0.0in}
\setlength{\parskip}{0.1in}
\begin{document}

\begin{titlepage}
    \let\footnotesize\small
    \let\footnoterule\relax
    \let \footnote \thanks
    \setcounter{footnote}{0}
    \begin{center}
      \setlength{\parskip}{0pt}
       {\large School Of Electronics and Computer Science \par}
      {\large Faculty of Physical and Applied Sciences \par}
      {\large University of Southampton \par}
      \vspace{29mm}
      {\large Argyris Zardilis \par}	
	\vspace{4mm}
      \large \today
	\vspace{13mm}
        \center
        {\Large \bf Tool for parameter inference in dynamic biological systems \par}
        \vspace{60mm}
      {\large Project Supervisor: Dr. Srinandan Dasmahapatra \par }
      {\large Second Examiner: Dr. Markus Brede \par}
      \vspace{12mm}
        {\large A progress report submitted for the award of }\\
      {\large BSc Computer Science }
    \end{center}
    \vfil\null
  \end{titlepage}
\begin{abstract}
Recent advances in experimental technology have given us detailed and comprehensive information on networks of biological interactions governing various functions of living organisms. This had led to an increase in the use of theoretical mathematical models to describe dynamic biological systems which has in turn led to an increasing need for computational tools to assist in the process of constructing these models and estimating their parameters from available experimental data. Although there is a rich literature on parameter estimation using a number of different techniques, very few attempts have been made to systematically attack the parameter estimation problem for these kinds of systems. From those, almost all of them attempt a mere reproduction of experimental data, disregarding important properties of those systems like their qualitative features and the effect of global system dynamics. 

In this study a computational tool has been produced tackling the parameter estimation problem in dynamic biological systems using different tecnhiques and its success has been tested with real world models. Further to that, an attempt has been made to uncover the link and interplay between the practical considerations of the parameter inference process and the more theoretical tools of sensitivity and bifurcation analysis and therefore consequently to systems dynamics which are captured and understood through these. 
\end{abstract}
\tableofcontents
\newpage
\section{Introduction}


\section{Background}

\section{Work}

\subsection{Methods}

\subsection{Results}

\section{Discussion and Future Work}

\newpage
\bibliography{progress}
\appendix
\section{Time Management}
\section{Critical Evaluation}
\end{document}