\documentclass[12pt,a4paper,titlepage]{article}
\usepackage{fullpage}
\usepackage{hyperref}
\usepackage[pdftex]{graphicx}
\bibliographystyle{plain}
\newcommand{\HRule}{\rule{\linewidth}{0.5mm}}
\usepackage{url}
\usepackage{amsmath}
\usepackage{enumitem}
\usepackage{float}
\usepackage{graphicx}
\usepackage{caption}
\usepackage{subcaption}
\usepackage{tabularx}
\usepackage[nottoc]{tocbibind}
\setlength{\parindent}{0.0in}
\setlength{\parskip}{0.1in}
\begin{document}

\begin{titlepage}
    \let\footnotesize\small
    \let\footnoterule\relax
    \let \footnote \thanks
    \setcounter{footnote}{0}
    \begin{center}
      \setlength{\parskip}{0pt}
       {\large School Of Electronics and Computer Science \par}
      {\large Faculty of Physical and Applied Sciences \par}
      {\large University of Southampton \par}
      \vspace{29mm}
      {\large Argyris Zardilis \par}	
	\vspace{4mm}
      \large \today
	\vspace{13mm}
        \center
        {\Large \bf Tool for parameter inference in dynamic biological systems \par}
        \vspace{60mm}
      {\large Project Supervisor: Dr. Srinandan Dasmahapatra \par }
      {\large Second Examiner: Dr. Markus Brede \par}
      \vspace{12mm}
        {\large A progress report submitted for the award of }\\
      {\large BSc Computer Science }
    \end{center}
    \vfil\null
  \end{titlepage}
\begin{abstract}
The increase in the use of theoretical mathematical models to describe biological systems has led to an increasing need for computational tools to assist in the process of constructing those models and estimating their parameters from available experimental data. Although there is a rich literature on parameter estimation using a number of different techniques, very few attempts have been made to produce computational tools that systematically attack the parameter estimation problem. From those, almost all of them attempt to reproduce experimental data, disregarding qualitative features of the systems and other requirements we might have from the model arising from the dynamic behaviour of such systems as they evolve or respond to external or internal stimuli.

The aim of this project it to produce a computational tool for automatic parameter estimation in generic dynamic biological systems taking into account the dynamic behaviour of such systems.
\end{abstract}
\tableofcontents
\newpage
\section{Introduction}


\section{Background}

\section{Work}

\subsection{Methods}

\subsection{Results}

\section{Discussion and Future Work}

\newpage
\bibliography{progress}
\appendix
\section{Time Management}
\section{Critical Evaluation}
\end{document}